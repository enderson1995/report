\begin{center}
    \setcounter{section}{2}
    \section*{CAPÍTULO II}
    \addcontentsline{toc}{section}{PLANTEAMIENTO DEL PROBLEMA}
    \vspace*{0.5in}
    \textbf{TEMA DE INVESTIGACIÓN}
\end{center}

\subsection{Planteamiento del problema}
%    que es. para que es. y su uso?\\

    Los recursos de generación de energía eléctrica son elementos finitos, sin embargo,  con el transcurso del tiempo la demanda se 
    hace cada vez mayor; en consecuencia se eleva el costo de la energía en proporción al incremento del consumo. La facturación eléctrica
    en distintos países dependen de la oferta y la demanda variando el precio
    en función de las horas en las cuales se genera mayor consumo. Estas variaciones en el costo del servicio generan una 
    oportunidad de ahorro a través de un habito de consumo inteligente.\\
    
    Tradicionalmente las empresas que proveen energía eléctrica calculan los costos con vatímetros analógicos que miden el 
    consumo en una casa, edificio o zona en general. Para obtener la información de este instrumento de medición es 
    requerido que un operario lea la medida implicando costos operativos, además de errores de transcripción o de lectura
    por parte del mismo.\\

    En los últimos años se ha venido desarrollando el Internet de las Cosas (Internet of Things, IoT por sus siglas en inglés),
    %    aqui puedes mejorar el concepto de IoT, esto es la conexión de dispositivos y sensores (inteligentes o no, lo que implica Inteligenia Artificial), a una red internet, de un concepto breve tipo este que estoy escribiendo en este comentario, derivas el proveccho que mencionas; \\
    
    %    En los ultimos años se ha venido desarrrollando el Internet de las Cosas (Internet of Things, IoT por sus siglas en inglés), una tecnología que permite la interconexión de objetos a través de dispositivos y sensores en una red hacia Internet, facilitando la transmisión de la información en forma continua e identificada de los instrumentos de medición en un lugar determinado, además, ejercer control sobre los mismos. ” \\
    Dentro de ella surge la inmotica, la cual se encarga de la automatización integral de
    edificaciones, permitiendo el monitoreo y control del mismo. Al emplear herramientas que posibiliten registrar parámetros de consumo es     factible generar un histórico que facilite a los usuarios finales plantear estrategias de consumo inteligente a corto, mediano y largo plazo para el consumo de la energía, permitendo así reducir costos asociados al consumo de la energía.
    
    % quizas puedes explicar aqui que tiene que ver o la relación de la inmotica, automatización  con un SCADA, o responder: ¿en el caso de montar un sistema de monitoreo para energía, que sería ese sistema y cual sería como que su propósito?  y tazz le mentes le ligas el concepto de SCADA pero enfocado a lo que planteas//
    %Los sistemas SCADA (Supervisory Control And Data Acquisition) son soluciones de software que plantean la adquisición y 
    %control de industrias a distancias. Al desarrollar este tipo de herramienta es posible crear interfaces capaces de 
    %mostrar la información relevante a los usuarios finales permitiendo así establecer planes en base al consumo.\\

    En zonas residenciales se requiere que la empresa que suministra la energía elétrica tenga un historico en tiempo real 
    del comportamiento de consumo de sus usuarios y a su vez que estos usuarios tengan un historico asociado a su consumo 
    que le permita planificar su consumo en función de las horas de menor costo. 
    
    % Aqui podrías definir bien las características del objeto de estudio de las pasantias, se trata de un edificio residencial que tiene 10 piso, dos apartamentos por piso, no sé alguna información sobre el objeto de estudio y/o hablar un poco sobre el sistema de energía que vas a monitorear, //
        
    En función de todo lo antes expresado, la pasantía tiene como objeto el Diseno de interfaz de un SCADA para el monitoreo del consumo energético de un edificio residencia a través de Mango Automation el cual es una herramienta de software que permite desarrollar interfaz para el control y monitorieo de cualquier entorno. 
     
     % creo que serría bueno que especificaras en alguna parte que el sistema de monitoreo contemplará las variables tal y tal, para no seguir hablando de forma tan general, pero no estoy segura de si quede bien o no, si quieres puedes probar //
     
\subsection{Objetivos General}
    Diseñar interfaz SCADA para el 
    monitoreo del consumo energético de un 
    edificio residencial.

\subsection{Objetivos Específicos}
\begin{itemize}
    \item Estudiar funcionalidades y configuración
    del software Mango Automation para el diseño de 
    interfaces.
    \item Identificar variables asociadas al sistema.
    \item Diseñar estructura de las interfaz.
    \item Desarrolarr interfaz asociada al sistema.
\end{itemize}
\newpage
