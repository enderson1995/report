\begin{center}
    \setcounter{section}{2}
    \section*{CAPÍTULO II}
    \addcontentsline{toc}{section}{PLANTEAMIENTO DEL PROBLEMA}
    \vspace*{0.5in}
    \textbf{TEMA DE INVESTIGACIÓN}
\end{center}

\subsection{Planteamiento del problema}
%    que es. para que es. y su uso?\\

    Los recursos de generación de energía eléctrica son elementos finitos, sin embargo,  con el transcurso del tiempo la demanda se 
    hace cada vez mayor; en consecuencia se eleva el costo de la energía en proporción al incremento del consumo. La facturación eléctrica
    en distintos países dependen de la oferta y la demanda variando el precio
    en función de las horas en las cuales se genera mayor consumo. Estas variaciones en el costo del servicio generan una 
    oportunidad de ahorro a través de un habito de consumo inteligente.\\
    
    Tradicionalmente las empresas que proveen energía eléctrica calculan los costos con vatímetros analógicos que miden el 
    consumo en una casa, edificio o zona en general. Para obtener la información de este instrumento de medición es 
    requerido que un operario lea la medida implicando costos operativos, además de errores de transcripción o de lectura
    por parte del mismo.\\

    En los últimos años se ha venido desarrollando el Internet de las Cosas (Internet of Things, IoT por sus siglas en inglés),
    el cual es la conexión de dispositivos inteligentes o no a la red internet permitiendo recolectar data y tomar desiciones 
    con la mínima interacción humana. Se desarrollan áreas como inmotica que a través de sensores y dispositivo se automatiza, controla
    y monitorea edificaciones, que integrados a sistemas SCADA costituye todo un sistema.\\

    %El IoT tecnología que se ha venido desarrollando en los últimos años, aprovecha el desarrollo en el área de las 
    %comunicaciones para transmitir información en forma continua e identificada de los instrumentos de medición en un
    %lugar determinado. Dentro de ella surge la inmotica el cual se encarga de la automatización integral de
    %edificaciones, permitiendo el monitoreo y control del mismo.\\
    
    Al emplear herramientas que permitan registrar parámetros de consumo es posible generar un histórico que permitan 
    platear estrategias a corto, mediano y largo plazo para el consumo de la energía.\\

    %Los sistemas SCADA (Supervisory Control And Data Acquisition) son soluciones de software que plantean la adquisición y 
    %control de industrias a distancias. Al desarrollar este tipo de herramienta es posible crear interfaces capaces de 
    %mostrar la información relevante a los usuarios finales permitiendo así establecer planes en base al consumo.\\

    En zonas residenciales se requiere que la empresa que suministra la energía elétrica tenga un historico en tiempo real 
    del comportamiento de consumo de sus usuarios y a su vez que estos usuarios tengan un historico asociado a su consumo 
    que le permita planificar su consumo en función de las horas de menor costo. Para establecer así el consumo inteligente 
    por parte de los usuarios finales, permitendo así reducir costos asociados al costo de la energía.\\
        
    La pasantía se enfoca en el monitoreo de un edificio residencial a través de Mango Automation el cual es una herramienta
    de software que permite desarrollar interfaz para el control y monitorieo de cualquier entorno.

\subsection{Objetivos General}
    Diseñar interfaz SCADA para el 
    monitoreo del consumo energético de un 
    edificio residencial.

\subsection{Objetivos Específicos}
\begin{itemize}
    \item Estudiar funcionalidades y configuración
    del software Mango Automation para el diseño de 
    interfaces.
    \item Identificar variables asociadas al sistema.
    \item Diseñar estructura de las interfaz.
    \item Desarrollar interfaz asociada al sistema.
\end{itemize}
\newpage